\documentclass[11pt]{article}

\usepackage{amsmath,amssymb,amsthm,setspace,tabto,fancyhdr,sectsty,mathtools}
\usepackage{titleps}
\usepackage[left=1.00in,right=1.00in,top=0.75in,bottom=1.50in]{geometry}
\usepackage{graphicx}
\graphicspath{ {assets/note3/} }

% start pdfinlimg (GPLv3, https://github.com/zerotoc/pdfinlimg/blob/master/pdfinlimg.sty)
\newcommand{\pdfinlimg}[5]{
\makebox[#1cm][l]{\immediate\pdfliteral{
  q
  #3 0 0 #4 0 0 cm
  #1 0 0 #2 0 0 cm
  0.885 0 0 0.885 0 0 cm 
  BI
  /W #3
  /H #4
  /CS /RGB
  /BPC 8
  /F [ /AHx /Fl ]
  ID
  #5>
  EI
  Q
}\vbox to #2cm{}}
}
% end pdfinlimg

% BEGIN PARAGRAPH STUFF
\usepackage[utf8]{inputenc}
\usepackage[english]{babel}
 
\setlength{\parindent}{4em}
% \setlength{\parskip}{1em}
\renewcommand{\baselinestretch}{1}
% END PARAGRAPH STUFF

% useful commands
\DeclarePairedDelimiter{\ceil}{\lceil}{\rceil}
\DeclarePairedDelimiter{\floor}{\lfloor}{\rfloor}

\newpagestyle{footers} {
    \sethead{}{}{}
    \setfoot{\small Intro to Crypto and Blockchain, Note \notenum}{\thepage}{\small Lin, Akhtar}
    \footskip = 45pt
}

\fancypagestyle{firstpage} {
    \vspace*{3\baselineskip}
    \footskip = 0pt
    \renewcommand{\headrulewidth}{6pt}
    \chead{\rule{\textwidth}{6pt} \vspace{20pt}\\}
    \lhead{\setstretch{1.05}\Large\fontfamily{lmdh}\selectfont
    Introduction to Cryptocurrencies and Blockchain 
    \\ Lin, Akhtar}
    \rhead{\huge \fontfamily{lmdh}\selectfont    Note \notenum}
    \lfoot{\small Intro to Crypto and Blockchain, Note \notenum}
    \rfoot{\small Lin, Akhtar}
}
    
\sectionfont{\Large\fontfamily{lmdh}\selectfont}

% for initial paragraph indent
\usepackage{indentfirst}

% UPDATE THIS FOR EVERY NEW NOTE
\newcommand{\notenum}{3}

\pagestyle{footers}

\begin{document}
    \thispagestyle{firstpage}
    \vspace*{2\baselineskip}
    \section*{Wallets and Cryptography}
    
    For the average user, before conducting any serious transactions, often the first step is to download wallet software to help manage funds. This note is dedicated to explaining the various types of bitcoin wallets, as well as the underlying cryptography that keeps everyone safe.
    
    \section*{Bitcoin vs Gmail Address Creation}
    
    As a brief review of the public and private keys in Bitcoin, we can compare them with that of Gmail. When generating a public/private key pair to use in Bitcoin, the user first randomly generates a private key, which is then used to calculate a public key. The set of possible private keys is so large that it is cryptographically safe to just generate a key. In contrast, the average user registering for an email account with Gmail would choose an address (username) and a password. Username and password are independent of each other in this case, and are not generated from each other, as in Bitcoin. The user would then send a request to the web server hosting , and submit this request to a central registry or mail server to check if the address is available. The main difference in the generation of public and private keys in both examples is centralization. In Bitcoin, users trust the underlying security of cryptography and the negligible probability of colliding public keys. Meanwhile, users of Gmail trust Google to safely store their private information on their servers. 
    
    \section*{Base 58}
    
    Conventionallly 
    
    
    % BEGIN KEY TERMS
    \newpage
    \thispagestyle{firstpage}
    \vspace*{2\baselineskip}
    \section*{Key Terms}
    \noindent A collection of terms mentioned in the note which may or may not have been described. Look to external sources for deeper understanding of any non-crypto/blockchain terms.
    \begin{enumerate}
        % edit within here
        \item \textbf{VOCAB WORD} --- Definition. % format
    \end{enumerate}
    % END KEY TERMS
\end{document}