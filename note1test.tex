\documentclass[full.tex]{subfiles}

\begin{document}

    \thispagestyle{firstpage}
    \vspace*{3\baselineskip}


    \section*{Bitcoin History: From the Cypherpunk Movement to JPMorgan Chase}
    Over the years, Bitcoin demonstrated its unique prowess as a currency, \textbf{store of value}, and more. As an open financial network secured with \textbf{public/private key encryption}, Bitcoin allows for \textbf{pseudonymous} financial transactions. Thanks to the Internet, Bitcoin is easily accessible and adoptable, allowing for censorship-resistant payments across borders.
    
    In the modern day, Bitcoin can be used to send money cheaply and efficiently across political and geographic borders. As a ``borderless" currency, Bitcoin enables users to send money that retains its value in every jurisdiction, particularly useful in developing economies and as an overseas payment mechanism. Bitcoin also demonstrates immense potential for digital trading, machine to machine payments (especially regarding the \textbf{Internet of Things}), autonomous networks, micropayments, and as an alternative store of value. In this note, we will explore the history of Bitcoin and the sequence of events that brought its existence. The goal is to provide insight into the backgrounds and motivations behind the present-day leaders of the Blockchain industry.
    
    \section*{Pre Bitcoin-2009: Libertarian Dreams and Ideals}
    Rapid advancements in computing technology in the 1980s and '90s brought along debates of computing rights and privacy. Rooted in libertarianism and cryptography, a community of dissenters who called themselves the ``\textbf{Cypherpunks}" began an anti-government movement to counter increasing surveillance enabled by evolving technologies. Eric Hughe's ``A Cypherpunk's Manifesto" states: ``Privacy is necessary for an open society and in the electronic age."
    
    Cypherpunks speculated about technology's role in shaping interactions between the individual and the state. They believed that individuals deserved access to these new tools, had the right to protect their personal information, and argued against government attempts to deteriorate any individual's privacy.
    
    An early attempt at establishing a currency separated from the state went by the name of \textbf{DigiCash}, an early cryptocurrency. Seeing the existing financial system as one of the greatest threats to individual privacy, cryptographer \textbf{David Chaum}, inventor of Blind Signature Technology, implemented DigiCash using the latest advancements in public and private key cryptography. Digicash promised complete privacy for users conducting online transactions and included a system of cryptographic protocols, preventing banks and governments from tracing personal online payments. However, a fatal flaw rested at the heart of Digicash's design: centralization. Chaum's company bore the overwhelming burden of validating every digital signature in the system, leading to bankruptcy in 1998. For the community, DigiCash represented not so much a failure as an important learning opportunity. \textit{Moral of the story: For a cryptocurrency to be stable, it must be \underline{decentralized}.}
    
    \section*{2009-2010: Early Development of Bitcoin}
    
    In October 2008, a \textbf{whitepaper} titled \textit{Bitcoin: A Peer-to-Peer Electronic Cash System} appeared online, published by an unknown person (or persons) by the pseudonym \textbf{Satoshi Nakamoto}. His nine-page white paper seemed to combine all previous efforts to create a self-sustaining digital currency. Although some were disheartened by historical attempts (such as DigiCash) to implement a cryptocurrency, a few early pioneers quickly recognized Bitcoin's potential and began supporting the project.
    
    Within a year, Satoshi and other developers released the first Bitcoin implementation. The \textbf{Genesis block} was mined in January 3, 2009. The \textbf{coinbase} of this block references a story in the Times of London newspaper involving the Chancellor bailing out banks. The first bitcoin transaction occurred on January 12, 2009. A year later on May 21, 2010, an individual first accepted Bitcoin in exchance for a tangible asset. Laszlo Hanyecz purchased \$25 worth of pizza for 10,000 BTC. Little did he know that 10,000 BTC would be worth about \$20,169,600 (as of May 20, 2017).
    \section*{2010-2012: Scandals, Hacks, Illegal Activity}
    In the early stages of Bitcoin, there existed no exchange for trading between bitcoin and regular currencies. In 2010, the creator of online service \textbf{mtgox.com} (\textit{\underline{M}agic: \underline{T}he 
    \underline{G}athering \underline{O}nline} e\underline{X}change, where \emph{Magic: The Gathering Online} players traded cards like stocks) decided to convert the domain into a bitcoin exchange website. Mt.\ Gox consolidated itself as the biggest bitcoin exchange during the beginning stages of Bitcoin. Since then, Mt.\ Gox suffered various security breaches and thefts. On June 19, 2011, a hacker allegedly used a Mt.\ Gox auditor's computer to transfer an abundance of bitcoins to himself. Mt. Gox was subsequently shut down for seven days to investigate the situation and received several lawsuits, some of which remain unresolved.
    
    ``\textbf{Silk Road}," another Bitcoin-centric site, also developed during this time. An illicit marketplace for drug deals nicknamed ``The eBay for drugs," Silk Road generated a significant impact in the Bitcoin ecosystem, constituting a huge portion of daily transactions, as most illegal merchants only accepted bitcoins due to Bitcoin's anonymous and untraceable features. After launching in February 2011, Silk Road was shut down in October 2013 by the FBI, who seized 3.6 milllion USD worth of bitcoin. Ross Ulbricht, the founder of Silk Road, was issued a life sentence without possibility of parole. The highly publicized story of Silk Road's take-down also changed general sentiment towards Bitcoin.
    
    Association of Bitcoin with illegal online activity, scandals, and hacks swayed public opinion against the cryptocurrency. Despite negative views, speculation about Bitcoin's future popularity also drove the price of one bitcoin to over one thousand USD. For better or for worse, the flurry surrounding Bitcoin drew an unprecedented amount of attention.
    \section*{2013-2014: Bitcoin Attracts Attention}
    
    Just four months after the FBI revealed and shut down revealing the illegal bitcoin marketplace Silk Road, Mt.\ Gox declared for bankruptcy in February 2014. Mt. Gox reported losing 744,408 BTC in a theft that went unnoticed for years.
    
    In March of 2014, CoinDesk, a popular online Bitcoin news platform, published an article claiming that Bitcoin inventor Satoshi Nakamoto had been ``found" in California. The man in question denied his association with the name.
    
    As Bitcoin entered the mainstream market, renowned venture capitalist Tim Draper announced his purchase of nearly 30,000 BTC. In an interview, Draper was optimistic that the price of bitcoin would rise to 10,000 USD; as he sees it, ``an entire economy is being rebuilt." In an interview, Draper described bitcoin as \textbf{bullish} in emerging markets, and encouraged investors to buy bitcoin.
    
    2014 also marked the year when major retailers and corporations began accepting bitcoin. In January, Overstock.com and Porn.com became the first companies to accept bitcoin. In April, new marijuana vending machines in Colorado intended on accepting bitcoin. In September, PayPal partnered with Coinbase and Bitpay, two large exchanges. In December, Microsoft also began accepting bitcoin payments.
    \section*{2013-2014: Venture Funded Bitcoin Startups} 
    
    Hype for Bitcoin fueled a rise in venture funded Bitcoin startups. One of the most notable was Coinbase, a digital asset exchange company based in San Francisco. Coinbase served mainly as a centralized, online bitcoin wallet, also serving as an important tool for investors to gauge interest in Bitcoin. Founded in June 2012, Coinbase witnessed drastic increases in funding in response to Bitcoin hype. In May 2013, Coinbase received a 5 million USD \textbf{Series A}; December 2013, 25 million USD \textbf{Series B}; July 2015, 75 million USD \textbf{Series C}.
    
    Other early venture funded Bitcoin startups included BitFinex, 21 Inc, BitPay, and Blockstream. Around this time, Andreessen Horowitz (a16z) also got involved in the Bitcoin space. Andreessen Horowitz has been directly involved in developing the Bitcoin industry, as they began investing in Bitcoin Core (the group that maintains the core Bitcoin software), sidechains, and various research firms in the space.
    
    Hype for Bitcoin seemed to die down in 2014, demonstrated by a plummet in bitcoin price. Clearly, Bitcoin was not in its best state.
    
    \section*{2015-Present: Ethereum Blows Up, Bitcoin Bounces Back, ``Enterprise Blockchain"}
    
    Between 2015 and the present (May 2017), a great number of initiatives developed, ranging from other cryptocurrencies to non-crypto blockchain use cases and startups. \textit{(Disclaimer: Not an exhaustive list.)}
    
    A whitepaper released in late 2013 by Russian programmer and cofounder of Bitcoin Magazine described the premise of a Turing-complete protocol running on a blockchain. (Bitcoin has a scripting language with basic functionality but nothing powerful enough to handle complex logic.) In July 2015, this idea came to fruition in form of the Ethereum blockchain, which serves as a platform for people to build and deploy decentralized applications. Applications on Ethereum require ``fuel" in the form of Ether tokens to run (more on this in a later note.) Less than a year later, Ether tokens were valued at a total of one billion USD. Ethereum represented a new future for decentralized technologies, and inspired such ideas as new governance models.
    
    Political events in 2016 and early 2017 also had a profound impact on the growing popularity of Bitcoin and blockchain technologies. Most notably, Brexit, India's war on cash, and the inauguration of President Donald Trump in the United States were political events that caused a number of individuals to lose trust and turn away from their governments in favor of decentralized and trustless alternative: Bitcoin. Additionally, economic events in the media highlighting the circumvention of capital controls, as well as general instability in the market caused the price of bitcoin to increase dramatically.
    
    There has also a growing interest in blockchain technologies from banks. Although big banks previously disregarded Bitcoin due to a lack of understanding and resentment towards decentralized control, several banks recently demonstrated interest in integrating blockchain technology into existing services. As Chairman, President, and CEO of JPMorgan Chase, Jamie Dimon stated in February 2016: ``The Blockchain is a technology, which we've been studying \ldots and yes it's real. It could probably reduce the cost of real application in certain things \ldots If it proves to be cheap and secure it will be adopted for a whole bunch of stuff.'' The separation of Bitcoin and blockchain in the discourse of even big banks reflects the importance of these technologies in recent past.
    
    From the underground Cypherpunk movement to one of the largest and most powerful centralized banking institutions in the United States, Bitcoin leaves permanent marks on industries and continues to rise into mainstream popularity. The culmination of the cutting edge in computer science, cryptography, and the passionate support by a dedicated community has elevated Bitcoin and blockchain technologies to a spotlight that is on its way to profoundly reinvent the future.
    
    \newpage
    \thispagestyle{firstpage}
    \vspace*{3\baselineskip}
    \section*{Key Terms}
    \noindent A collection of terms mentioned in the note which may or may not have been described. Look to external sources for deeper understanding of any non-crypto/blockchain terms.
    \begin{enumerate}
        \item \textbf{Bullish} --- In reference to stocks, having the expectation to rise in value.
    
        \item \textbf{Coinbase} --- (Not the exchange!) The ``coinbase" (conventionally lowercase, not a proper noun) is a part of the Bitcoin blockchain where a ``miner" (not yet formally introduced term) sends a block reward to themselves as a pat of the back for securing the network. In the context of this note: the coinbase has a section in which a miner can leave a small amount of arbitrary information which the rest of the network will save but ignore, similar to comments in code. Most people choose to leave this field blank, as they have no need for it. Satoshi decided to use it to store a link that is now included in the first block of every (valid) version of the Bitcoin blockchain.
        
        \item \textbf{Cypherpunks} --- A group of 1980s and '90s anti-establishment, libertarian activists whose loyalty lies primarily in code and cryptography. Huge inspiration for privacy-centric cryptocurrencies.
        
        \item \textbf{David Chaum} --- Considered the ``Father of Anonymity" by some, he received a doctorate in computer science (at Berkeley!), going on to develop DigiCash in an attempt to create an electronic, anonymous currency separate from the state.
        
        \item \textbf{DigiCash} --- A landmark attempt at creating an electronic, anonymous currency. Failed due to lack of scalability, as one entity alone could not monitor and verify the volume of digital signatures.
        
        \item \textbf{Genesis block} --- The first block of a blockchain. A term originally coined for Bitcoin, the first block mined of any blockchain is now referred to as the ``genesis block" for that particular blockchain.
        
        \item \textbf{Internet of Things (IoT)} --- Connecting various electronic devices that were not originally intended for interoperability. 
        
        \item \textbf{Mt.\ Gox (mtgox.com)} --- A classic Japanese \textit{Magic:\ The Gathering} card trading site turned bitcoin exchange story, this exchange incepted in 2011 became infamous for lost funds, hacks, poor management, and dozens of lawsuits. While skepticism about Bitcoin increased, the name spread as well. (A similar gist to Jack Sparrow's retort to being the worst pirate ever heard of: ``But you have \emph{heard} of me." -- \textit{Nadir})
        
        \item \textbf{Pseudonymous} --- While accounts are allegedly anonymous, experts can trace these random online accounts to real people with enough information, meaning that the accounts can be deanonymized with \underline{thorough investigation}. Difficult, but not impossible.
        
        \newpage % for convenience, not necessity; do change with judgment
        
        \item \textbf{Public/private key encryption} --- In the context of cryptocurrencies, a method by which users certify ownership of a particular account or quantity of coins. Every user has at least one public/private key pair. The public key is given out to others for receiving money, the private key is used to ``unlock" those funds.
        
        \item \textbf{Satoshi Nakamoto} --- The pseudonym for the creator of the Bitcoin whitepaper. Many suspects put forward, but no conclusion on the true identity. (I think it's some British person, from Satoshi using ``bloody" in GitHub commits to putting punctuation outside of quotation marks -- \textit{Nadir})
        
        \item \textbf{Series A/B/C} --- Rounds of funding for startups. Each one demonstrates potential and viability for the company's success, and at each state certain indicators have various weights.
        
        \item \textbf{Silk Road} --- An illegal drug marketplace run by Ross Ulbricht, who called himself ``Dread Pirate Roberts," transactions were primarily conducted in Bitcoin. Started in 2011, shut down by FBI in 2013. While previous Silk Road employees attempted to start Silk Road 2.0, FBI once again shut down the site.
        
        \item \textbf{Store of value} --- An asset presumed to maintain its value or appreciate (increase in value) over time. Gold, a classic example.
        
        \item \textbf{Whitepaper} --- A formal academic proposal, defended with research and reasoning.
    \end{enumerate}
\end{document}