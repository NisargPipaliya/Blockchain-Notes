\documentclass[full.tex]{subfiles}


% change this line
\graphicspath{ {assets/note8/} }


\providecommand{\notenum}{8}


\begin{document}
    \thispagestyle{firstpage}
    \vspace*{2\baselineskip}
    \section*{Alternative Consensus and Enterprise Blockchain}
    
    At the most fundamental level, all public blockchains rely on efficient and secure consensus algorithms. In our analysis of Bitcoin and Ethereum, we learned how the Proof-of-Work algorithm allows both networks to come to consensus on blocks. Consensus algorithms in general ensure that the network on which it is deployed agrees on a singular blockchain as the only truth, and that malicious users cannot successfully break or fork the blockchain. In study, we find that Proof-of-Work wastes enormous amounts of energy for computation, may lead to centralization where electricity (for mining) is cheap, and in general does not scale well. This note will discuss some alternative consensus protocols, and how they are used both in public and private blockchains.
    
    \section*{Properties of a Distributed System}
    
    Recall that a state is a condition at a specific period of time. Think the state of your bank account, and the balance included in it. Think the state of the Ethereum network, and the balances of all the accounts in Ethereum. In order to ensure correctness of state, distributed systems must have two essential properties.
    
    The first essential property is \textbf{liveness}, which says that the current state of a system satisfies liveness if there is some path from the current state to some future state where the liveness property also holds true. In simpler terms, the liveness property guarantees that a network will always make progress, and that something good will happen eventually. In the context of consensus algorithms, liveness guarantees that all nodes will eventually decide on a value, whih ultimately defines a state. (Note the use of the word \textit{decide} and \textit{agree}, because a node may have to conform to what is agreed upon by the rest of the network.)
    
    The second essential property is \textbf{safety}, which says that the current state guarantees the safety property if \textit{all} the future states that can be reached from the current state also satisfies the safety property. In simpler terms, the safety property guarantees that something bad will never happen. In regards to consensus algorithms, safety guarantees that no two nodes will decide on different values.
    
    In the \textbf{Fischer Lynch Paterson (FLP) Impossibility Theorem}, there are a total of three fundamental properties of distributed systems. We introduced liveness and safety already. The third property is fault tolerance, which guarantees that if one node in a system fails, the entire node will not fail. FLP Impossibility states that in any consensus system, only two of the three properties can be satisfied at once. Although fault tolerance is important, we will focus mainly on how to ensure safety and liveness in this note.
    
    \section*{Distributed Properties in Consensus}
    
    In an asynchronous distributed system, we cannot guarantee decisions (safety) and correct decisions (safety) at the same time. You can only guarantee eventual liveness or eventual consistency. This means that if no new updates are made to a data item, eventually all access to that item will return the last updated value. Everyone sees the same thing. In terms of consensus on a blockchain, eventual consistency means that all nodes in a network will eventually share a common view of the state of the blockchain. Keep in mind that blockchains are immutable --- blocks can only be appended to the end of a blockchain, and that it is difficult to alter the middle of a blockchain. 
    
    \section*{Key Consensus Protocol Problems}
    
    Two key problems that consensus protocols must tackle are Sybil attacks and the Byzantine Generals Problem. Recall that Sybil attacks involve flooding a network with false identities and transferring malicious information to other nodes. Bitcoin makes Sybil attacks difficult because of its underlying Proof-of-Work algorithm and also some design considerations, including restrictions on connecting with geographically close IP addresses.
    
    The Byzantine Generals Problem is a classic agreement problem that involves a group of Byzantine generals surrounding a city, each in a different location. 
    
    
    
    
    % BEGIN KEY TERMS
    \newpage
    \thispagestyle{firstpage}
    \vspace*{2\baselineskip}
    \section*{Key Terms}
    \noindent A collection of terms mentioned in the note which may or may not have been described. Look to external sources for deeper understanding of any non-crypto/blockchain terms.
    \begin{enumerate}
        % edit within here
        \item \textbf{VOCAB WORD} --- Definition. % format
    \end{enumerate}
    % END KEY TERMS
\end{document}